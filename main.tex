%-------------------------
% Resume in Latex
% Author : Jake Gutierrez
% Based off of: https://github.com/sb2nov/resume
% License : MIT
%------------------------

\documentclass[letterpaper,11pt]{article}
\usepackage[scale=0.85]{geometry}
\usepackage{graphicx}
\usepackage{latexsym}
\usepackage[empty]{fullpage}
\usepackage{titlesec}
\usepackage{marvosym}
\usepackage[usenames,dvipsnames]{color}
\usepackage{verbatim}
\usepackage{xcolor}
\usepackage{enumitem}
\usepackage{hyperref}
\usepackage{apacite}
\usepackage{fancyhdr}
\usepackage[spanish]{babel}
\usepackage{tabularx}
\input{glyphtounicode}


%----------FONT OPTIONS----------
% sans-serif
% \usepackage[sfdefault]{FiraSans}
% \usepackage[sfdefault]{roboto}
% \usepackage[sfdefault]{noto-sans}
% \usepackage[default]{sourcesanspro}

% serif
% \usepackage{CormorantGaramond}
% \usepackage{charter}


\pagestyle{fancy}
\fancyhf{} % clear all header and footer fields
\fancyfoot{}
\renewcommand{\headrulewidth}{0pt}
\renewcommand{\footrulewidth}{0pt}

% Adjust margins
\addtolength{\oddsidemargin}{-0.5in}
\addtolength{\evensidemargin}{-0.5in}
\addtolength{\textwidth}{1in}
\addtolength{\topmargin}{-.5in}
\addtolength{\textheight}{1.0in}

\urlstyle{same}

\raggedbottom
\raggedright
\setlength{\tabcolsep}{0in}

% Sections formatting
\titleformat{\section}{
  \vspace{-4pt}\scshape\raggedright\large
}{}{0em}{}[\color{black}\titlerule \vspace{-5pt}]

% Ensure that generate pdf is machine readable/ATS parsable
\pdfgentounicode=1

%-------------------------
% Custom commands
\newcommand{\resumeItem}[1]{
  \item\small{
    {#1 \vspace{-2pt}}
  }
}

\newcommand{\resumeSubheading}[4]{
  \vspace{-2pt}\item
    \begin{tabular*}{0.97\textwidth}[t]{l@{\extracolsep{\fill}}r}
      \textbf{#1} & #2 \\
      \textit{\small#3} & \textit{\small #4} \\
    \end{tabular*}\vspace{-7pt}
}

\newcommand{\resumeSubSubheading}[2]{
    \item
    \begin{tabular*}{0.97\textwidth}{l@{\extracolsep{\fill}}r}
      \textit{\small#1} & \textit{\small #2} \\
    \end{tabular*}\vspace{-7pt}
}

\newcommand{\resumeProjectHeading}[2]{
    \item
    \begin{tabular*}{0.97\textwidth}{l@{\extracolsep{\fill}}r}
      \small#1 & #2 \\
    \end{tabular*}\vspace{-7pt}
}

\newcommand{\resumeSubItem}[1]{\resumeItem{#1}\vspace{-4pt}}

\renewcommand\labelitemii{$\vcenter{\hbox{\tiny$\bullet$}}$}

\newcommand{\resumeSubHeadingListStart}{\begin{itemize}[leftmargin=0.15in, label={}]}
\newcommand{\resumeSubHeadingListEnd}{\end{itemize}}
\newcommand{\resumeItemListStart}{\begin{itemize}}
\newcommand{\resumeItemListEnd}{\end{itemize}\vspace{-5pt}}

%-------------------------------------------
%%%%%%  RESUME STARTS HERE  %%%%%%%%%%%%%%%%%%%%%%%%%%%%


\begin{document}

%----------HEADING----------
% \begin{tabular*}{\textwidth}{l@{\extracolsep{\fill}}r}
%   \textbf{\href{http://sourabhbajaj.com/}{\Large Sourabh Bajaj}} & Email : \href{mailto:sourabh@sourabhbajaj.com}{sourabh@sourabhbajaj.com}\\
%   \href{http://sourabhbajaj.com/}{http://www.sourabhbajaj.com} & Mobile : +1-123-456-7890 \\
% \end{tabular*}

\textbf{\Large \scshape \textcolor{black}{Colegio de Estudios Superiores de Administración - CESA}}\\
Undergraduate Program in Business Administration\\
    \\ \vspace{1pt}
Emphasis in Advanced Analytics\\
Syllabus - Course \# 01882 - 2022-1; Semester: 2021-2.\\
Academic Coordinator: Alvaro Moncada, Ph.D. $|$ Email: \href{mailto:amoncada@cesa.edu.co}{\textcolor{blue}{amoncada@cesa.edu.co}} \\
Principal Professor: Juan C. Correa, Ph.D. $|$ Email: \href{mailto:juan.correan@cesa.edu.co}{\textcolor{blue}{juan.correan@cesa.edu.co}} \\
Assistant Professor: Nicolás Gomez-Osoario, MsSc $|$ Email: \href{mailto:ngomezo@cesa.edu.co}{\textcolor{blue}{ngomezo@cesa.edu.co}} \\


\begin{picture}(0,0)
\put(465,30){\includegraphics[width=2.5cm]{OL.png}}
\end{picture}

\section{1. General Information}
\begin{table}[h!]
\centering
\begin{tabular}{|ll|}
\hline
\multicolumn{2}{|l|}{~~Name of the Course:  Business Data Analytics}                                                                                                                                                                                        \\ \hline
\multicolumn{1}{|l|}{~~Code: 1882}                                                                                                                             &~Prerequisites: Probability, Statistics, Computational Modeling~~~                                                                    \\ \hline
\multicolumn{1}{|l|}{\begin{tabular}[c]{@{}l@{}}~~Total of Class Hours: 48\\~~Total of Autonomous Learning Hours: 96~~\\ ~~Total Learning Experience Hours: 144~~ \end{tabular}} & \begin{tabular}[c]{@{}l@{}}~Schedule: Mondays \& Thursday\\ ~5:30 a 7:00 pm\\~90 minutes class\end{tabular} \\ \hline
\end{tabular}
\end{table}

\section{2. Policies}
\textbf{Attending, Academic Fraud \& Communication}:
\begin{itemize}
\item[a] In the event that an evaluation activity will be lost by a student, he or she is obliged to notify a valid excuse within a term not exceeding eight (8) calendar days following the date of his/her absence. If alternative evaluation activities are validated by the head of the program, the professor is allowed to reschedule evaluation conditions and dates.
\item[b] Any student will be graded with zero if they do not hand in their activities on time. Exceptions will be handled following the preceding point.
\item[c] Class hours begin and end following the scheduled plan. The professor is allowed to authorize or deny delayed entrance or anticipated leave to classes.
\item[d] The official means for communication, grading, and academic feedback purposes is the institutional learning managemt system (LMS).
\item[e] During this course, contentwise materials will be illustrated via research-based evidence and business cases. Sensitive data will be protected when local or copyright regulations apply.
\item[f] Because there are no fixed or standard approaches applicable for all cases in business data analytics problems, rubrics are not used to guide students' performance during this course. Instead, best data analytics practices are continuously explained in teaching sessions, so students can identify how to apply them for evaluation activities. 
\end{itemize}

\section{Content-based Modules per Week}
\begin{itemize}
\item \textbf{Week 1}. Welcoming Session (session 1), Simulator Presentation (session 2)
\item \textbf{Week 2}. Business Data Analytics Simulator (session 3 and 4)
\item \textbf{Week 3}. Business Data Analytics Toolbox: Anaconda (session 5) and GitHub Desktop (session 6)
\item \textbf{Week 4}. Data Sources, Data sets (session 7) and Hierarchy Pyramid (session 8)
\item \textbf{Week 5}. Data Visualization (session 9) and \textbf{First Partial Exam} (session 10)
\item \textbf{Week 6}. Twitter API as data source (session 11) and Web Scraping (session 12)
\item \textbf{Week 7}. Data Manipulation Tasks (session 13 and 14)
\item \textbf{Week 8}. Applied Data Mining Methods (session 15 and 16)
\item \textbf{Week 9}. Case Study 1: Online Food Delivery Services (session 17 and 18)
\item \textbf{Week 10}. Case Study 2: Trust-Sales in Latin American E-Commerce (session 19 and 20)
\item \textbf{Week 11}. \textbf{Second Partial Exam} (session 21) Exam feedback (session 22)
\item \textbf{Week 12}. Business Analytics in the real world (session 23) and Guests from the Industry (session 24)
\item \textbf{Week 13}. Guests from the Industry (session 25) and Case Study 3: Political Marketing (session 26)
\item \textbf{Week 14}. Case Study 4: Disposition to Use Automated Baking Services (session 27), Final Exam Preparation (session 28)
\item \textbf{Week 15}.\textbf{Final Exam} (session 29 and 30)
\item \textbf{Week 16}. \textbf{Final Exam} (session 31) and Final exam feedback (session 32)
\end{itemize}

\section{3. Evaluation Plan}
\begin{table}[h!]
\centering
Evaluated activities \& weights\\
\begin{tabular}{|l|c|}
\hline
~~Activity &~~\multicolumn{1}{l|}{Percentage~~} \\ \hline
~~Partial Exam 1~~&~~15                              \\ \hline
~~Partial Exam 2~~&~~15                              \\ \hline
~~Workshops~~&~~50                              \\ \hline
~~Final Exam~~&~~20                              \\ \hline
\end{tabular}
\end{table}

\begin{table}[h!]
\centering
Evaluation Criteria
\begin{tabular}{|l|l|l|}
\hline
Evaluation Activity~~&~~Description &~~Criteria\\ \hline
Partial exams    & \begin{tabular}[c]{@{}l@{}} These are individual activities \\ intended to be done in a 90-minutes session\\ These activities evaluate conceptual knowledge~~\\ and computational ability to conduct \\ business-oriented data analyses \\ emphasizing technical rigor and \\effective communication.\end{tabular} & \begin{tabular}[c]{@{}l@{}}These activities are done\\  inside or outside the learning management \\ system as timely indicated by\\  the professor. They should reveal the technical \\ use of libraries, packages, and/or software\\with a clear description of the context where the\\ data comes from with the format specified by \\the instructions of the evaluations.\end{tabular}                                                                   \\ \hline
Workshops     & \begin{tabular}[c]{@{}l@{}}These activities assess competencies \\ (not conceptual knowledge) that illustrate \\ individual's ability to apply knowledge \\ for using technologies taught in classes \\ with the same criteria of partial exams.\end{tabular}                                         & \begin{tabular}[c]{@{}l@{}}All workshops will be done during teaching \\ sessions, unless otherwise informed \\ Handing in conditions will be informed\\ in the instruction of each workshop.\end{tabular} \\ \hline
Final Exam~~ & \begin{tabular}[c]{@{}l@{}}The final exam is the combination of\\ partial exams and workshops and it\\ encompasses all contents taught \\during the course\end{tabular} & \begin{tabular}[c]{@{}l@{}}The final exam is a comprehensive evaluation\\ covering all contents during the course.\end{tabular}                                \\ \hline
\end{tabular}
\end{table}

\newpage
\begin{itemize}
\item[1] The course is approved with a minimum grading of \textbf{6.0 out of 10.0}. The maximum possible grading is \textbf{10.0}.
\item[2] Partial exams and workshops will be done individually or in teams as informed by the professor. They are previously scheduled and communicated at the begining of the semester. \textbf{Students are not exempted from the final exam} because this is designed to assess the global knowledge and ability that the student gained during the course.
\item[3] The difficulty of evaluation activities will be increased as a function of time and student progress during the course. 
\end{itemize}


\section{4. References}
\begin{itemize}
\item Correa, J.C. \& Camargo, J. (2017). Ideological Consumerism in Colombian Elections: Links between Political Ideology, Twitter Activity and Electoral Results. \textit{Cyberpsychology, Behaviour, and Social Networking}, \textit{20}(1), 37 – 43. doi: DOI: 10.1089/cyber.2016.0402
\item Correa, J.C., Garzón, W., Brooker, P., Sakarkar, G., Carranza, S., Yunado, L. \& Rincón, A. (2019). Evaluation of Collaborative Consumption of Online Food Delivery Services through Web Mining Techniques. \textit{Journal of Retailing and Consumer Services}, \textit{46}, 45-50. DOI:10.1016/j.jretconser.2018.05.002
\item Correa, J. C., Laverde-Rojas, H., Martínez, C. A., Camargo, O. J., Rojas-Matute, G. \& Sandoval-Escobar, M. (2021). The Consistency of Trust-Sales Relationship in Latin American E-Commerce. \textit{Journal of Internet Commerce}. (Online First). DOI: 10.1080/15332861.2021.1975426
\item Correa, J.C., Dakduk, S., van der Woude, D., Sandoval-Escobar, M. \& López, R. (2022). Low-Income Consumers’ Disposition to Use Automated Banking Services. \textit{Cogent Business \& Management}, \textit{9}(1), 2071099 DOI: 10.1080/23311975.2022.2071099
\item Laverde-Rojas, H. \& Correa, J.C. (2019). Can Scientific Productivity Impact the Economic Complexity of Countries? \textit{Scientometrics, 120}(1), 267-282. 
\item Molin, S. (2021). \textit{Hands-on Data Analysis with Pandas} (2nd Edition). Birmingham: Packt Publishers
\item Schwarz, J. S., Chapman, C., \& Feit, E. M. (2020). \textit{Python for marketing research and analytics}. Cham, Switzerland. Springer Nature. 
\item Segura, M. A. \& Correa, J.C. (2019). Data of Collaborative Consumption in Online Food Delivery Services. \textit{Data in Brief, 25}, 104007. DOI: 10.1016/j.dib.2019.104007
\item Shmueli, G., Bruce, P. C., Gedeck, P., \& Patel, N. R. (2020). \textit{Data mining for business analytics: Concepts, techniques, and applications in Python}. New Jersey, USA: Wiley & Sons 
\item Teichert, T., Rezaei, S. \& Correa, J.C. (2020). Customers’ experiences of fast food delivery services: Uncovering the semantic core benefits, actual and augmented product by text mining. \textit{British Food Journal}, \textit{122}, (11), 3513-3528. DOI: 10.1108/BFJ-12-2019-0909
\item Zins, C. (2007). Conceptual approaches for defining data, information, and knowledge. \textit{Journal of the American Society for Information Science and Technology}, \textit{58}(4), 479–493. DOI: 10.1002/asi.20508
\end{itemize}




\section{5. Online Resources}
\textcolor{blue}{\url{https://github.com/stefmolin/Hands-On-Data-Analysis-with-Pandas-2nd-edition}}

\textcolor{blue}{\url{https://github.com/jcorrean/mypythonadventure}}

\textcolor{blue}{\url{https://www.kaggle.com/}}

\textcolor{blue}{\url{https://rda.ucar.edu/}}

\textcolor{blue}{\url{https://datadryad.org/stash}}

\textcolor{blue}{\url{http://archive.eso.org/cms.html}}

\textcolor{blue}{\url{https://www.dataone.org/}}

\textcolor{blue}{\url{https://figshare.com/}}

\textcolor{blue}{\url{https://www.icpsr.umich.edu/web/pages/}}

\textcolor{blue}{\url{https://www.re3data.org/}}

\textcolor{blue}{\url{https://ciser.cornell.edu/data/data-archive/}}

\textcolor{blue}{\url{https://dataverse.harvard.edu/}}

\textcolor{blue}{\url{https://www.europeansocialsurvey.org/}}

% \bibliographystyle{apacite}
% \bibliography{refs}
\end{document}